\section{Obiettivi}

L'obiettivo principale del presente elaborato è di migliorare alcune caratteristiche del sistema di annotazione WATSS, utilizzando alcune tecniche di Computer Vision al fine di agevolare la creazione di annotazioni.

\subsection{Interfaccia utente}

Da un'analisi del sistema precedente, emergeva che alcune operazioni effettuabili tramite interfaccia risultavano essere poco intuitive per l'utilizzatore. In particolare, dai risultati del \emph{System Usability Scale (SUS)} presentato in \cite{Bartoli:2015:WWA:2733373.2807411}, si ritiene che non sia molto semplice imparare ad usare il sistema.

Le modifiche all'interfaccia grafica sono state dunque apportate con il fine di rendere più chiaro per l'utente le varie funzioni messe a disposizione dal sistema, a partire dalla schermata iniziale devono essere chiare fin da subito le sue caratteristiche e potenzialità.

\subsection{Creazione e modifica delle annotazioni}

La parte fondamentale del sistema è la fase di creazione e raffinamento delle bounding box all'interno della scena. Queste sono definite mediante una serie di rettangoli associati ad opportuni metadati. Per questo motivo, deve essere semplice ed immediato per l'utente poter interagire con le annotazioni, modificandole e inserendole senza difficoltà. 

Il precedente sistema presenta alcune difficoltà, non consentendo una rapida modifica all'utente, rendendo l'azione di inserimento delle annotazioni leggermente complicata e difficile da gestire.

Obiettivo per questo aspetto è quello di introdurre un nuovo sistema di creazione e modifica delle annotazioni.

\subsection{Timeline}

Una delle caratteristiche mancanti in questo tool, presenti invece in molti altri sistemi di annotazione, è una \emph{timeline}. Questa ha come scopo principale quello della navigazione tra i vari frames e la visualizzazione temporale delle annotazioni inserite. 

Con questo strumento è infatti possibile visualizzare la durata di permanenza di una stessa persona in più frames consecutivi, consentendo all'utente di avere maggiore controllo sulla annotazioni inserite ed andare a correggere eventuali mancanze. 

\subsection{Predizione delle annotazioni}

Dato il gran numero di frame da annotare, può risultare molto utile avere a disposizione un sistema di \emph{predizione} delle annotazioni future in base ad una selezione corrente. 
Nel sistema è implementato un semplice meccanismo di predizione che ripropone una stessa bounding box nel frame successivo che può essere \emph{approvata} con un click da parte dell'utente.

Mediante tecniche di Computer Vision si vuole fornire dei \emph{proposals} per la posizione e la dimensione della stessa persona nei frame successivi. La predizione verrà valutata mediante la combinazione di più tecniche, come ad esempio la stima del moto, un \emph{pedestrian detector} ed una stima mediante filtro di \emph{Kalman}.

La fase di generazione dei proposals deve integrarsi nell'interfaccia, in particolar modo nella timeline.

\subsubsection{Geometria della scena}

Un altro tipo di predizione può essere effettuata inoltre conoscendo la geometria della scena. Questo è reso possibile se è nota la calibrazione delle telecamere con cui sono stati scattati i frames e se le telecamere sono fisse.

Utilizzando le informazioni spaziali è possibile ad esempio prevedere l'altezza di una persona data la sua posizione nella scena, consentendo così, ad esempio, di ridimensionare automaticamente la bounding box in base alla posizione in cui si vuole inserire.


