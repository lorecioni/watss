\section{Conclusioni}

In questo elaborato è presentato il lavoro di miglioramento dell'applicazione WATSS, un sistema web per creazione di dataset annotati dagli utenti.

In seguito ad un'analisi comparativa di altri sistemi di annotazioni e del sistema in esame, sono state proposte ed apportate alcune modifiche volte al miglioramento dell'esperienza utente nel suo utilizzo. In particolare è stata posta maggiore attenzione in:
\begin{itemize}
\item Interfaccia grafica utente
\item Inserimento e modifica delle annotazioni inserite
\item Esportazione dei dati e configurazioni del sistema
\item Generazione di \emph{proposals} utilizzando tecniche di \emph{object tracking}
\item Utilizzo delle informazioni derivate dalla conoscenza della \emph{geometria della scena}
\item Aggiornamento della base dati
\item Introduzione di una pagina di installazione guidata
\end{itemize}

Le modifiche apportate sono state verificate con esperimenti di tipo qualitativo.